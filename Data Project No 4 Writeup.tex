\documentclass[12pt]{article}
 
\usepackage[margin=1in]{geometry} 
\usepackage{amsmath,amsthm,amssymb}
\usepackage{graphicx}
 
\begin{document}
 
\begin{title}
{Data Project No 4: The Implementation and Application of the Hierarchical Risk Parity Algorithm for Asset Allocation:}
\end{title}
\begin{author}
{\textbf{Name:} Vignesh Iyer
\\ \textbf{Collaborators:} Bruce Englert, Sarah Marshall, Madison Shaerr, \\and Zach Shepelak}
\end{author}
\maketitle

\begin{enumerate}

% question 1
\item{Paper Discussion: \textbf{\textit{BUILDING DIVERSIFIED PORTFOLIOS THAT OUTPERFORM OUT-OF-SAMPLE}}}
%Mean-variance portfolios are optimal for a training set, but for a testing set they tend to not perform well.
\\Markowitz portfolio theory falls short when the attempt is made to compute the optimal point on the Markowitz Efficient Frontier via the covariance matrix. The condition number gets larger when adding correlated investments, sometimes too large to the extent in which the errors create instability for the inverse matrix, or ill conditioning. 

As the investments hold stronger correlation, there will be a greater need for portfolio diversification. Though, this increase in diversification will lead to more instability.

The minimum of $\frac{1}{2}N(N+1)$ independent and identically distributed (IID) observations are needed in order to estimate an $N \times N$ non-singular covariance matrix. This is problematic as correlation structures do not remain unchanged over the time period required to gather such a large number of observations.

Outside sources attempt to create a more robust method which include adding more constraints, adding Bayesian priors, and improving the stability of the inverse covariance matrix.

The Hierarchical Portfolio Construction (HRP) method uses the covariance matrix but does not need positive-definiteness or inversion of the covariance matrix.

\textbf{Steps for Hierarchical Portfolio Construction:}
\\1. Tree Clustering:
Stock observations are grouped by their correlation matrices. We create a tree graph by computing the Euclidean distance between all column vectors and cluster columns based on this data. Linkage criterion is further used to refine the clustering. 
\\2. Quasi-diagonalization: 
We use matrix diagonalization to reorganize the covariance matrix to place highly correlated stocks together. This allows for optimal weight distribution after inverse-variance allocation.
\\3. Recursive bisection:
We use recursive bisection based on cluster covariance on the quasi-diagonal matrix to determine optimal allocation.

\newpage
\item\textbf{Formation of the HRP portfolio and comparison with the other optimal portfolios listed in the package:}
\begin{verbatim}
Long-only Maximum Sharpe Portfolio (discrete weights):
Expected annual return: 111.3%
Annual volatility: 15.1%
Sharpe Ratio: 7.25
20 out of 28 tickers were removed
Funds remaining: 167.88
Discrete allocation: {'EBAY': 102, 'PG': 19, 'PEP': 16, 'SNAP': 73,
'ORCL': 12, 'ROKU': 6, 'FB': 1, 'NFLX': 0}
Funds remaining: $167.88

Long-only Minimum Volatility Portfolio (weight cap, not discrete weights):
{'AAPL': 0.02793345126703132, 'TSLA': 0.03390547917195495,
'NFLX': 0.024168442567386074, 'GOOG': 0.03541871725139596,
'INTC': 0.033958562163552274, 'BA': 0.03617042326824796,
'MSFT': 0.03702119373376347, 'WMT': 0.05606026246968804,
'FB': 0.03334387510268078, 'PEP': 0.05066735492135562, 
'JNJ': 0.049969761927294194, 'PG': 0.050884260376100554,
'NVDA': 0.01116622991123388, 'KO': 0.051865447608405384,
'DIS': 0.04969664173244947, 'EBAY': 0.048046948054210704,
'AMZN': 0.029680729620693246, 'HPQ': 0.03017356800113899,
'AMD': 2.4462311516704194e-18, 'VZ': 0.052943742220931365,
'T': 0.0519557723803737, 'SNAP': 0.025360924595998386,
'ORCL': 0.04506524299359991, 'APC': 0.03235515666849452,
'TXN': 0.03232526723942072, 'QCOM': 0.04562272746364013,
'ROKU': 0.0025069771958217244, 'TWTR': 0.02173284009313669}
Expected annual return: 55.5%
Annual volatility: 13.8%
Sharpe Ratio: 3.87

Long/Short Portfolio maximizing return for target volatility of 10%:
Expected annual return: 68.2%
Annual volatility: 10.0%
Sharpe Ratio: 6.62

Market-neutral Markowitz Portfolio finding the minimum volatility (target return 
of 20%):
Expected annual return: 20.0%
Annual volatility: 2.4%
Sharpe Ratio: 7.59



Custom Objective:
Expected annual return: 333.0%
Annual volatility: 61.1%
Sharpe Ratio: 5.42


Expected annual return: 326.1%
Annual volatility: 56.7%
Sharpe Ratio: 5.72

CVaR Optimization:
{'AAPL': 0.0, 'TSLA': 0.00024, 'NFLX': 0.0, 'GOOG': 0.0, 'INTC': 0.0004,
'BA': 0.00745, 'MSFT': 0.00622, 'WMT': 0.0319, 'FB': 0.0, 'PEP': 0.00152,
'JNJ': 0.01122, 'PG': 0.02994, 'NVDA': 0.0, 'KO': 0.11298, 'DIS': 0.04784,
'EBAY': 0.30814, 'AMZN': 0.00429, 'HPQ': 0.00689, 'AMD': 0.0, 'VZ': 0.34056,
'T': 0.05281, 'SNAP': 0.0035, 'ORCL': 0.01001, 'APC': 0.01585, 'TXN': 0.0,
'QCOM': 0.0, 'ROKU': 0.00622, 'TWTR': 0.00196}

Hierarchical Risk Parity Portfolio:
{'AAPL': 0.008157092313495545, 'AMD': 0.002129639602698393,
'AMZN': 0.009566141773722786, 'APC': 0.009434894538153306,
'BA': 0.028489309727744168, 'DIS': 0.11797386040169647,
'EBAY': 0.060645986787024306, 'FB': 0.011213094825353551,
'GOOG': 0.02484305045780576, 'HPQ': 0.013255964633319885,
'INTC': 0.028107444613527117, 'JNJ': 0.07446192200122966,
'KO': 0.03420300920630387, 'MSFT': 0.023398820510367468,
'NFLX': 0.008160991531781564, 'NVDA': 0.0041616439446699105,
'ORCL': 0.04403658469659668, 'PEP': 0.06764603349189276,
'PG': 0.08164449488731151, 'QCOM': 0.05292237901091357,
'ROKU': 0.0023679270311393126, 'SNAP': 0.0050199760557718695,
'T': 0.07434791874378803, 'TSLA': 0.008509891639229168,
'TWTR': 0.007062055600790238, 'TXN': 0.02105938616162691,
'VZ': 0.048927209934771065, 'WMT': 0.12825327587727509}
\end{verbatim}
\newpage
\item{\textbf{Discussion of Simulated Results}}
\\During the simulation, many of the stocks that were removed were deemed as the bad stock but this is inconsistent because what may be a bad stock in one portfolio might be better in another. The sharpe ratio calculations above demonstrates that better returns have been made based on the risk for each stock. Based on this scenario, the expected annual returns above also demonstrated that a good stock portfolio was selected.
\end{enumerate}

\end{document}